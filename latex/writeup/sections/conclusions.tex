\section{Conclusion}

In conclusion, the nonuniform FFT is a particularly appealing tool in the fields of image and signal processing.
However, identifying a fast and accurate method has been a significant challenge for more than half a century.
Kernel methods which smooth delta spikes of nonuniform information have emerged as the favorite method of which we explored the Gaussian, Kaiser-Bessel, and exponential of semicircle versions.
These each present unique levels of smooth decay in real and Fourier space which affects the accuracy and speed of the resulting transformation.
Additionally, we explored a low-rank DFT version which relies on snapping nonuniform points to uniform points via Chebyshev polynomials.
Finally, we presented some simple numerical experiments comparing the exponential of semicircle smoothing kernel against the traditional interpolation then FFT procedure.
In general, this area is still being actively researched, and we were able to recognize the challenge of balancing speed and accuracy.